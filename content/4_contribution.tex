\newcommand{\prev}{\diamond}

\section{Applying Ptolemy's Inequality to the Elkan's Algorithm}
\label{sec:contrib}

As seen in the previous section,
Elkan's algorithm can improve the execution speed of $k$-means clustering through the use of the triangle inequality.
Explicit distance calculations are avoided when tight upper and lower bounds of the distance function are available.

In this section, we use Ptolemy's inequality to calculate an additional set of bounds and modify Elkan's algorithm accordingly.
These Ptolemaic bounds often improve upon the triangular bounds, leading to increased execution speeds, as will be shown in Section \ref{sec: results}.


\subsection{Ptolemaic Bounds}
Recall that the triangle inequality relates the distances between three points,
while Ptolemy's inequality does so for four points.
At the end of an iteration of Elkan's algorithm (Step~5~to~7 in Alg.~\ref{alg:elkan}),
the distance bounds between the data points $x_i$ and the new center positions $c'_j$ are calculated.
Therefore, the three points used in the calculation of the bounds are the aforementioned two points and $c_j$, the cluster center calculated in the previous iteration.

A natural extension to this procedure is then to use an even older cluster center $c^\prev_j$ as a fourth point,
i.e. the cluster center calculated in the penultimate iteration.
To formalize, Ptolemy's inequality can be used to calculate the following upper and lower bounds:

\begin{theorem}[Ptolemaic bounds for Elkan's algorithm]
	During an iteration of Elkan's algorithm,
	let $x_i \in D \subset \mathbb{R}^n$ be a point from the dataset to be clustered,
	$d$ the Euclidean distance function,
	$c'_j$ a cluster center calculated in this iteration,
	$c_j$ a cluster center calculated in the previous iteration, and
	$c^\prev_j$ a cluster center calculated in the iteration before the previous iteration.
	A lower and upper bound on $d(x_i,c'_i)$ is then given by
	\begin{align}
		\label{eq:pto_upper}
		d(x_i, c_j') & \leq u_i' = \frac{1}{d(c_j, c_j^{\prev})} \cdot \left( u_i \cdot d(c_j', c_j^{\prev}) + u_i^{\prev} \cdot d(c_j', c_j) \right) \\
		\label{eq:pto_lower}
		d(x_i, c_j') & \geq l_{i,j}' = \frac{1}{d(c_j, c_j^{\prev})} \cdot \max \left\{
		\begin{array}{l}
			l_{i,j}^{\prev} \cdot d(c_j, c_j') - u_i \cdot d(c_j', c_j^{\prev}) \\
			l_{i,j} \cdot d(c_j', c_j^{\prev}) - u_i^{\prev} \cdot d(c_j, c_j')
		\end{array}
		\right\}\;,
	\end{align}
	where $u_i, u^\prev_i$ ($l_{i,j}, l^\prev_{i,j}$) are the lower (upper) bounds calculated in the previous and the penultimate iteration, respectively\footnote{
 For readability, we suppress the bounds' arguments in favor of indices, e.g. $u_i$ instead of $u(x_i)$.
    }.
\end{theorem}
\vspace{-2\baselineskip}
\begin{proof}
	As all points are in the $\mathbb{R}^n$ and the distances function is also Euclidean, Ptolemy's inequality
	$d(x, y)\cdot d(v, u) \leq d(x, v) \cdot d(y,v) + d(x, u) \cdot d(y, v)$ holds.
	Eq.~\ref{eq:pto_upper} can easily be shown by rearranging Ptolemy's inequality and
	noting that the upper bounds $u_i, u^\prev_i$ used in place of exact distances can only increase the right-hand side of the inequality. Thus, $d(x_i, c_j') \leq u_i'$ holds.

	The process is similar for Eq.~\ref{eq:pto_lower}; as the signs on the right-hand side are not equal in this case, there are two distinct ways to substitute into Ptolemy's inequality.
	This gives rise to two inequalities, of which the maximum is the stronger bound.
	Analogous to $u'_i$, the right-hand side of Eq.~\ref{eq:pto_lower} can only be made smaller by inserting lower bounds $l_{i,j},l_{i,j}^\prev$ in the minuend or upper bounds $u_i, u^\prev_i$ in the subtrahend.
	Thus, $d(x_i, c_j') \geq l_{i,j}'$ also holds.
\end{proof}


\subsection{Integration}
To integrate the novel bounds into the existing framework of Elkan's algorithm,
 Steps~6 and~7 in Alg.~\autoref{alg:elkan} need to be updated to use the new bounds.
It is possible to only replace either the upper or lower bound, which leads to a hybrid solution that is further explored in Section \ref{sec: results}.

As the Ptolemaic bounds require information from two previous iterations, the first iteration of the algorithm is always conducted with Elkan's original bounds.


% As Elkan's algorithm tries to avoid distance calculations during the assignment step,
% the bounds for the clusters–datapoint distances $d(x,c')$ are needed.
% Thus, two of the point in the triangle inequality are the the data point $x$ and the new cluster position, while the third point 
% As the majority of potential distance calulations are needed
% Elkan's algorithm distance computations during The lower and upper bounds 
% Two of these points have a 
% In Elkan's original algorithm, the three points 




